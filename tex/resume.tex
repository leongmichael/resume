%%%%%%%%%%%%%%%%%%%%%%%%%%%%%%%%%%%%%%%%%
% Medium Length Professional CV
% LaTeX Template
% Version 3.0 (December 17, 2022)
%
% This template originates from:
% https://www.LaTeXTemplates.com
%
% Author:
% Vel (vel@latextemplates.com)
%
% Original author:
% Trey Hunner (http://www.treyhunner.com/)
%
% License:
% CC BY-NC-SA 4.0 (https://creativecommons.org/licenses/by-nc-sa/4.0/)
%
%%%%%%%%%%%%%%%%%%%%%%%%%%%%%%%%%%%%%%%%%

%----------------------------------------------------------------------------------------
%	PACKAGES AND OTHER DOCUMENT CONFIGURATIONS
%----------------------------------------------------------------------------------------

\documentclass[
	%a4paper, % Uncomment for A4 paper size (default is US letter)
	11pt, % Default font size, can use 10pt, 11pt or 12pt
]{tex/resume} % Use the resume class

\usepackage{ebgaramond} % Use the EB Garamond font
\usepackage{hyperref}




%------------------------------------------------

\name{Michael Leong} % Your name to appear at the top

% You can use the \address command up to 3 times for 3 different addresses or pieces of contact information
% Any new lines (\\) you use in the \address commands will be converted to symbols, so each address will appear as a single line.
\address{mtleong@usc.edu $\vert$ \underline{\href{https://www.linkedin.com/in/m-leong/}{linkedin.com/in/m-leong}} $\vert$ \underline{\href{https://github.com/leongmichael}{github.com/leongmichael}} $\vert$ \underline{\href{https://leongmichael.github.io/}{leongmichael.github.io}}}% Contact information

%----------------------------------------------------------------------------------------

\begin{document}

%----------------------------------------------------------------------------------------
%	EDUCATION SECTION
%----------------------------------------------------------------------------------------

\begin{rSection}{Education}
	
	\textbf{University of Southern California} \hfill \textit{Los Angeles, CA} \\ 
	B.S. Electrical \& Computer Engineering \hfill \textit{}
	
\end{rSection}

%----------------------------------------------------------------------------------------
%	WORK EXPERIENCE SECTION
%----------------------------------------------------------------------------------------

\begin{rSection}{Experience}

	\begin{rSubsection}{Lawrence Berkeley National Laboratory}{June 2023 - Present}{Student Assistant}{Berkeley, CA}
		\item Led development of EV-CIPT (Electrical Vehicle Charging Infrastructure Planning Tool): full-stack web application that can estimate the requirements for charging infrastructure and the related electrical demands.
        \item Implemented control logic for fleet charging management, prioritizing vehicles based on state of charge (SoC), stay duration, and station availability.
        \item Utilized Express, React, Node webstack and GitHub Actions automated tests. Implemented REST API to communicate between client and Python server-side logic and simulations.
        \item Skills: React.js, Node.js, Express.js, REST API, NumPy, Pandas, GitHub Actions
        
	\end{rSubsection}

%------------------------------------------------

	\begin{rSubsection}{FRC 604 Quixilver Robotics}{September 2021 - June 2024}{VP of Engineering}{San Jose, CA}
		\item Led team in all branches of technical robot design. Responsible for training new members and contributing to robot development.
        \item Developed time-optimal trajectory optimization software utilizing non-linear optimization algorithms using our robot’s kinodynamic constraints.
        \item Developed particle filter robot localization software using both odometry and computer vision utilizing PhotonVision/OpenCV to detect AprilTag targets.
        \item Skills: OpenCV, CasADi, Matplotlib, NumPy
        
	\end{rSubsection}

%------------------------------------------------




\end{rSection}

%----------------------------------------------------------------------------------------
%	PROJECTS SECTION
%----------------------------------------------------------------------------------------

\begin{rSection}{Projects}

	\begin{rSubsection}{VelocityDraft}{}{GitHub Link: \underline{\href{https://github.com/shuklabhay/velocity-draft}{https://github.com/shuklabhay/velocity-draft}}}{}
		\item Flexible application essay writing scheduler. Third place in Onehacks III Hackathon against 120+ competitors.
		\item Leverages user writing abilities and essay deadlines to algorithmically determine best order of essay writing.
            \item Skills: Express.ts, React.ts, Node.js
	\end{rSubsection}

    \begin{rSubsection}{Canvas Final Grade Calculator}{}{GitHub Link: \underline{\href{https://github.com/leongmichael/canvas-final-grade-calculator}{https://github.com/leongmichael/canvas-final-grade-calculator}}}{}
		\item Web extension that fetches your Canvas grade and calculates the final exam score needed for your desired grade.
        \item 600+ users, 5.0 stars on Chrome Web Store.
            \item Skills: React.ts, Node.js

	\end{rSubsection}

%------------------------------------------------

\end{rSection}

%----------------------------------------------------------------------------------------
    %	SKILLS SECTION
%----------------------------------------------------------------------------------------

\begin{rSection}{Skills}

	\begin{tabular}{@{} >{\bfseries}l @{\hspace{6ex}} l @{}}
		Programming Languages & C++, Java, JavaScript, Python, TypeScript \\
		Libraries + Frameworks & Flask, Firebase, MongoDB, Express, React, Node, Matplotlib, NumPy, Pandas, SQL, TensorFlow \\
		Tools & Git, GitHub Actions, Onshape, Fusion360 \\
            Licenses + Certifications & DeepLearning.AI Machine Learning Specialization
	\end{tabular}

\end{rSection}


%----------------------------------------------------------------------------------------
    % HONORS & AWARDS SECTION
%----------------------------------------------------------------------------------------

\begin{rSection}{Honors \& Awards}

    \begin{itemize}
        \setlength\itemsep{-0.7em} % Adjust the space between items
        \item California Seal of Biliteracy for Chinese \hfill 2024
        \item FIRST Dean's List Semi-Finalist \hfill 2023
        \item OneHacks III Hackathon 3rd place submission \hfill 2023


    \end{itemize}

\end{rSection}

%----------------------------------------------------------------------------------------
%	EXAMPLE SECTION
%----------------------------------------------------------------------------------------

%\begin{rSection}{Section Name}

	%Section content\ldots

%\end{rSection}

%----------------------------------------------------------------------------------------

\end{document}
