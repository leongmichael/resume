%%%%%%%%%%%%%%%%%%%%%%%%%%%%%%%%%%%%%%%%%
% Medium Length Professional CV
% LaTeX Template
% Version 3.0 (December 17, 2022)
%
% This template originates from:
% https://www.LaTeXTemplates.com
%
% Author:
% Vel (vel@latextemplates.com)
%
% Original author:
% Trey Hunner (http://www.treyhunner.com/)
%
% License:
% CC BY-NC-SA 4.0 (https://creativecommons.org/licenses/by-nc-sa/4.0/)
%
%%%%%%%%%%%%%%%%%%%%%%%%%%%%%%%%%%%%%%%%%

%----------------------------------------------------------------------------------------
%	PACKAGES AND OTHER DOCUMENT CONFIGURATIONS
%----------------------------------------------------------------------------------------

\documentclass[
	%a4paper, % Uncomment for A4 paper size (default is US letter)
	11pt, % Default font size, can use 10pt, 11pt or 12pt
]{tex/resume} % Use the resume class

\usepackage{ebgaramond} % Use the EB Garamond font
\usepackage{hyperref}




%------------------------------------------------

\name{Michael Leong} % Your name to appear at the top

% You can use the \address command up to 3 times for 3 different addresses or pieces of contact information
% Any new lines (\\) you use in the \address commands will be converted to symbols, so each address will appear as a single line.
\address{408-710-0887 $\vert$ mtleong@usc.edu $\vert$ \underline{\href{https://www.linkedin.com/in/m-leong/}{https://www.linkedin.com/in/m-leong/}} $\vert$ \underline{\href{https://github.com/leongmichael}{https://github.com/leongmichael}}} % Contact information

%----------------------------------------------------------------------------------------

\begin{document}

%----------------------------------------------------------------------------------------
%	EDUCATION SECTION
%----------------------------------------------------------------------------------------

\begin{rSection}{Education}
	
	\textbf{University of Southern California} \hfill \textit{Expected May 2028} \\ 
	B.S. Computer Engineering \& Computer Science \hfill \textit{Los Angeles, CA}
	
\end{rSection}

%----------------------------------------------------------------------------------------
%	WORK EXPERIENCE SECTION
%----------------------------------------------------------------------------------------

\begin{rSection}{Experience}

	\begin{rSubsection}{Lawrence Berkeley National Laboratory}{July 2023 - Present}{Research Intern}{Berkeley, CA}
		\item Developed a web-based tool for assessing the building energy demand flexibility of small and medium-sized businesses.
		\item Set-up MongoDB, Express, React, Node (MERN) web stack and GitHub Actions automated tests.
        \item Ported existing building data to MongoDB and implemented REST API to communicate between server and client.
        \item Implemented and optimized existing demand flexibility algorithms to the application.
        
	\end{rSubsection}

%------------------------------------------------

	\begin{rSubsection}{FRC 604 Quixilver Robotics}{September 2021 - June 2024}{VP of Engineering}{San Jose, CA}
		\item Lead team in all branches of technical robot design. Responsible for training new members and contributing to robot development.
        \item Developed time-optimal trajectory optimization software utilizing non-linear optimization algorithms using our robot’s kinodynamic constraints.
        \item Developed particle filter robot localization software using both odometry and computer vision utilizing PhotonVision/OpenCV to detect AprilTag targets.
        
	\end{rSubsection}

%------------------------------------------------




\end{rSection}

%----------------------------------------------------------------------------------------
%	PROJECTS SECTION
%----------------------------------------------------------------------------------------

\begin{rSection}{Projects}

	\begin{rSubsection}{VelocityDraft}{}{GitHub Link: \underline{\href{https://github.com/shuklabhay/VelocityDraft}{https://github.com/shuklabhay/VelocityDraft}}}{}
		\item Created college essay writing scheduler based on writing speed, deadline, and number of necessary revisions. Created with Firebase, Express, React, Node (FERN) web stack.
		\item Received 3rd place submission while competing against 120+ people.
	\end{rSubsection}

    \begin{rSubsection}{Canvas Final Grade Calculator}{}{GitHub Link: \underline{\href{https://github.com/leongmichael/canvas-final-grade-calculator}{https://github.com/leongmichael/canvas-final-grade-calculator}}}{}
		\item Web extension that automatically fetches Canvas class grade and calculates the required score on your final exam to receive your desired grade.
        \item 100+ users, 5.0 stars on Chrome Web Store.
	\end{rSubsection}

%------------------------------------------------

\end{rSection}

%----------------------------------------------------------------------------------------
    %	SKILLS SECTION
%----------------------------------------------------------------------------------------

\begin{rSection}{Skills}

	\begin{tabular}{@{} >{\bfseries}l @{\hspace{6ex}} l @{}}
		Programming Languages & Java, JavaScript, Python, TypeScript \\
		Libraries + Frameworks & Flask, Firebase, MongoDB, Express, React, Node, React-Native, Expo, Matplotlib, Numpy \\
		Tools & Git, Onshape, Fusion360, Photoshop, Vegas Pro
	\end{tabular}

\end{rSection}


%----------------------------------------------------------------------------------------
    % HONORS & AWARDS SECTION
%----------------------------------------------------------------------------------------

\begin{rSection}{Honors \& Awards}

    \begin{itemize}
        \setlength\itemsep{-0.7em} % Adjust the space between items
        \item SVEC Education Scholarship Honorable Mention \hfill 2024
        \item California Seal of Biliteracy for Chinese \hfill 2024
        \item FIRST Dean's List Semi-Finalist \hfill 2023
        \item OneHacks III 3rd place submission \hfill 2023


    \end{itemize}

\end{rSection}

%----------------------------------------------------------------------------------------
%	EXAMPLE SECTION
%----------------------------------------------------------------------------------------

%\begin{rSection}{Section Name}

	%Section content\ldots

%\end{rSection}

%----------------------------------------------------------------------------------------

\end{document}
